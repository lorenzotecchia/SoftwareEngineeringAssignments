\chapter{Task 4}
\begin{parlist}
	\item  A sequence diagram is a diagram that describes the interaction of a user with software over time and is useful for refining the model in the requirements analysis and to explain the behaviour of the software to stakeholders.\cite{uwlaxInformalSemanticsSequence}
	\item To execute efficient modelling Sequence diagrams and Class diagrams usually complement each other, so they don't share any elements in common.
	\item In the vertical are shown the instances of the classes (Objects), occasionally the actor that starts the sequence of methods and the lifeline of the object itself. Also the activation of methods on objects are displayed vertically on the sequence diagram.
	\item Is represented by an horizontal arrow pointing to to the object that can perform said method. With the name of the method and its inputs
	\item Yes, The Messages are the Method calls these point with arrows to what the results of their execution are.
	\item The relation consist of the fact that the sender object has created the reciever object.
	\item You can deduce this from the fact that there is only one line and that only one specific order of events is depicted.
	\item Solving this point:
	\begin{enumerate}
		\item Sequence(method) depicted
		\item Participant 
		\item Actor
		\item General ordering
		\item Participant
		\item Lifeline
		\item Message
		\item Self Call
		\item Creation
		\item Creation
		\item Activation
		\item Return Message
		\item Optional Block
		\item Error message
		\item Deletion from other object
	\end{enumerate}
\end{parlist}

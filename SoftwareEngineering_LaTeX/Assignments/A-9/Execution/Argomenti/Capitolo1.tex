\chapter{Task 1}
\begin{parlist}
	\item In computer science, information hiding is the principle of segregation of the design decisions in a computer program that are most likely to change, thus protecting other parts of the program from extensive modification if the design decision is changed. Whereas the "need to know" principle states that: depending on the level of clearance that stakeholders have on the system, they access a certain amount of knowledge about the system itself. So for some level of knowledge about the system the information hiding principle get implemented for some users/stakeholders. \cite{wikipediaInformationHiding}
	\item The information hiding principle is useful when the design of a piece of software gets broken down into smaller parts, modules in this case. And different teams work on different modules of the same software. In this case the different teams do not need to know how other modules internally work, they just need to know how they can fit their module features into the whole system and/or with the other modules. A clear example of the information hiding principle would be how developer nowadays work with APIs and how they don't need to know or in most cases, they are not allowed to know how the systems behind those APIs work. It's like working with a black box. \cite{wikipediaInformationHiding}
	\item Design by contract prescribes that software designers should define formal, precise and verifiable interface specifications for software components, which extend the ordinary definition of abstract data types with preconditions, postconditions and invariants. These specifications are referred to as "contracts", in accordance with a conceptual metaphor with the conditions and obligations of business contracts.
	\item Invariants, postconditions, preconditions, are all constrains for the DbC principle. They are actuated by the OCL to improve the use of the Unified Modelling Language. So they act as an enhancement for UML in order to deliver a better understanding of the system that a UML diagram is depicting, usually a Class Diagram, because postconditions and preconditions are constrains for methods of classes and invariants are constraints for classes. \cite{wikipediaDesignContract}\cite{10.1007/978-3-642-30982-3_3}
\end{parlist}
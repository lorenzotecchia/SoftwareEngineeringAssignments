\chapter{Task 1}
\begin{elenco}
	\item Are identified the aspects: real-world goals, functions of constraints, software behaviour and to their evolution over time and across software families.\\ (Second chapter from line 5 to 11)
	\item The authors particularly mention the discipline of:
	\begin{itemize}
		\item Cognitive psychology
		\item Anthropology
		\item Sociology
		\item Linguistics
	\end{itemize}
	Also, mentioning an important philosophical element inside RE, the Authors identify also: epistemology, phenomenology and ontology as key aspects of the Requirement Engineer work. \\(Second page right side of the page)
	\item Theoretical computer science provides the framework to assess the feasibility of requirements, while practical computer science provides the tools by which software solutions are developed.\\(Second page third paragraph)
	\item The authors identify:
	\begin{itemize}
		\item Eliciting requirements: The act of requirement elicitation is directly correlated to the choice of modelling scheme, thus imprinting on the RE the correct elicitation technique to use. This would also mean choosing for the appropriate job the appropriate requirements to elicit, how and when implement these elicitation techniques and the formulation of elicitation process itself.
		\item Modelling and analysing requirements: Is the process of constructing abstract depiction that are susceptible to understanding. 
		\item Communicating requirements: It's the process in which the RE has to formulate an effective way of communicating the requirements identified; most of the time documenting said requirement in order to express to other stakeholders, either in formal or not formal language, all the important part identified by the RE. 
		\item Agreeing requirements: It's the uninterrupted process in which requirements passed through different stakeholders, often with different and/or divergent goals, should be agreed upon everybody. Often a very complicated process.
		\item Evolving requirements: The variations of the requirements must be recorded and managed in this key aspect of Requirement Engineering, mainly providing tools of version control and configuration management. Requirements are added in response to changing stakeholder needs, or because they were missed in the initial analysis.
	\end{itemize}(First page, first lines of the second column)
	\item The advantage would be that the RE doesn't need to focus more broadly on what possible solution of a problem would be, rather upper level of business requirements are refine into lower level technical requirement and so helping the RE to only focus on the domani of the problem and thus on the real need of the stakeholders.\\(Chapter 4.1, third paragraph)
	\item The contextual elicitation technique start on the premise on which local context are vital to fully understand social and organisational behaviour, and the observer must be inserted into this local context in order to gasp the way in which participants organise or create their social structures.\\(Chapter 4.2 latter part of the left part of the page)
	\item The authors mention:
	\begin{itemize}
		\item Enterprise Modelling: It's the modelling of organizations' structure. 
		\item Data Modelling: It's type of modelling done around large computer-based systems that often generate large amounts of data.
		\item Behavioural Modelling: Modelling done about the interaction between stakeholders, either future ones or present ones, regarding its dynamic form.
		\item Domain Modelling: The crucial act of constructing the abstract model in which the system will operate.
		\item Modelling (NFRs): The modelling of requirement that often cannot be validated.
	\end{itemize}(All the 5th chapter)
	\item The authors mark the similarities between the processes of requirement validation and proofing of scientific theories. Mainly that the act of requirement validation like scientific  proof should apply the same principle of refuting: it should devise experiments to attempt to refute the current statement of requirements.\\(Chapter 7, beginning of the right part of the page).
	\end{elenco}

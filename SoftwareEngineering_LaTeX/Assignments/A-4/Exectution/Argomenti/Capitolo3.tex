\chapter{Task 3}
\begin{elenco}
	\item These are the domain concepts identified:
	\begin{itemize}
		\item There are a number of people to assign to different floors.
		\item There is a counter per each floor
		\item Each counter is assigned to one person
		\item There are lunch breaks that imply break substitutions 
		\item There is a time in which new trainees must be assigned to supervisors, after they have completed their training they can sit alone everywhere except the ground floor.
		\item The main entrance which is on the ground floor provides WIFI tickets or new IDs.
		\item Employees can take sick days and communicate it to the person that is in charge of the roaster.
		\item Employees can express preference for turns either working during the morning shift or the afternoon shift. 
		\item The plan has to be shown to employees, preferably before the week starts. 
	\end{itemize}
	\item These are the questions identified:
	\begin{itemize}
		\item How trainees can see to which supervisor they are assigned?
		\item How can you see who's assigned to which break turn?
		\item How can turn preferences be registered by the person that does the turns?
		\item How can employees submit errors for the roaster?
		\item 
		\item
		\item
		\item
	\end{itemize}
	\item There are mainly three groups of people: The management, the employees and the trainees. The management needs to be notified promptly about the employees sick days, and if any error occurs into the roaster, must be able to amend these errors without doing all the work altogether. The employees must be notified before Monday morning about their work turns, and be able to send feedback about their preferences of work turns(morning or afternoon) and about any errors inside the roster itself. The trainees must be able to identify from the roaster the floor to which they've been assigned and their shift, consequently their supervisor. 
	\begin{itemize}
		\item 
		\item 
		\item 
	\end{itemize}
	\item 
	\item The survey technique mainly implemented the question and answer method. The main problem with this would be that each actor of the library (management/employee/trainee) has their own problems and requirements and often the contrasts between these parties can effect the effectiveness of the questionnaire. Also not gathering enough information from other stakeholders like the trainees could be a problem since the system wouldn't include their requirements into consideration. Maybe a contextual elicitation would be better in order to fully grasp the problems of all the parties, and take every problem into consideration, for example a period of time of a week in which the RE assist the management in doing the roaster and at the same time cover employees and trainees feedback to full formulate a satisfying system for every stakeholder.
\end{elenco}
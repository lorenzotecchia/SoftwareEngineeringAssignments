\chapter{Aufgabe 2-2}
Idee: 2D Physik simulation in Java.

Ideen: 2D Physik simulation in Java, Zeitplan app, app die Anzeigt welche büsse wann bzw. demnächst in deiner umgebung ankommen und wo diese fahren und piepst wenn du los musst,  Program zum erstellen und traversieren von accuraten 3D karten von echten Umgebungen, Programm zur berechnung der leistung eines electro motors,  Program zum Automatischen Löschung und findung von ungewollten dateien. Programm zur hilfe von schreiben von Beweisen. Programm für das Persönliche zeit management.  Whats app email functionalitäty addon, 

\section{a)}
2D Physik simulation in Java:
Es gibt das zwar schon http://www.jbox2d.org/ aber keine ahnung vielleicht zu kompliziert???
Das Projekt eine Bibliotheke die es einem ermöglichen möglichst einfach und schnell 2d simulationen in echtzeit in Java zu machen. Sodass Java-spiele entwickler schnell physik in ihere spiele einbauen können. Die entwickler die es nutzen sollten nicht viel lesen müssen um es zu benutzen. Und es kann auf einer schon existierenden Java simulations bibliothek die schwer zu verwenden ist vielleicht gebaut werden.

\section{???}

Was ist die Anwendungsdomäne der Software?
Die Anwendungs domaine ist die effektive Simulation von vereinfachter Physik in Java. 
Um welche Art Software handelt es sich, wozu wird sie benötigt?
Um eine Bibliotheke die Wiederverwendet werden kann und Physik berechnungen macht
– Welchen Arbeitsprozess soll sie unterstützen / verbessern?
Es sollte den Arbeitsprozess des Erstellens von 2D Speielen und cooler Animationen in Java unterstüzen.
– Welche Funktionen soll sie bereitstellen?
- Das definieren von 2D Formen und Objekten und die simulation von deren bewegungen als physische Objekte in einer 2D ebene.
– Im welchem Verhältnis stehen die Kosten zum Nutzen der Software?
• Wer sind Ihre Kunden (d.h. woher kommt der – gedachte – Auftrag und das Geld
für die Entwicklung)?
Die Kunden sind Spiele entwickeler und visulelle software designer die in Java arbeiten wollen
• Wer ist die Zielgruppe Ihrer Software?
Programmierer
Welcher Wissenstand im Umgang mit der Software ist von den Benutzern notwendig bzw. zu erwarten?
Einen sehr genauen und vollständigen Wissenstand
Welche Probleme / Widerstände bei der Einführung der Software sind zu erwarten?
Optimierungs probleme

Java Easy Library Manager for Windows

    
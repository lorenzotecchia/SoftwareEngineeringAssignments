\chapter{Task 2}
\begin{parlist}
 \item OCL of klassifiziereDreieck
 \begin{lstlisting}[language=OCL,frame=trBL]
context Dreieck::klassifiziereDreieck(seite1 : int, seite2 : int, seite3 : int) 
  pre: seite1 > 0 and seite2 > 0 and seite3 > 0 and not seite1.oclIsUndefined() and not seite2.oclIsUndefined() and not seite3.oclIsUndefined()
\end{lstlisting}
\item Test Cases in Java

\begin{center}
\begin{tabular}{ | c | c | } 
  \hline
  \begin{lstlisting}[language=java,frame=trBL]

@Test
public void testgleichseitig(){
   DreiecksArt result = klassifiziereDreieck(4,4,4);
   assertEquals("Pruefe ob es Gleichseitigkeit erkennt", DriecksArt.Gliechseitig, result);
}
  \end{lstlisting} text& Need to check if it can recognize equallateral triangles. \\ 
  \hline
  \begin{lstlisting}[language=java,frame=trBL]
@Test
public void testrechtwinklig(){
   DreiecksArt result = klassifiziereDreieck(3,4,5);
   assertEquals("Pruefe ob es Rechtwinkligkeit erkennt", DriecksArt.Rechtwinklig, result);
}
  \end{lstlisting} & Need to check if it can recognize right triangles. \\ 
  \hline
  \begin{lstlisting}[language=java,frame=trBL]
@Test
public void testgleichschenklig(){
   DreiecksArt result = klassifiziereDreieck(4,4,5);
   assertEquals("Pruefe ob es Gleichschenkligkeit erkennt", DriecksArt.Gleichschenklig, result);
}
  \end{lstlisting} &  Need to check if it can recognize isosceles triangles. \\ 
  \hline
  \begin{lstlisting}[language=java,frame=trBL]
@Test
public void testnulleingabe(){
   DreiecksArt result = klassifiziereDreieck(0,4,5);
   assertEquals("Pruefe ob es Schlechte Laengen erkennt", DriecksArt.Normal, result);
}
  \end{lstlisting} &  Need to check if it can recognize invalid lengths. \\ 
 \hline
  \begin{lstlisting}[language=java,frame=trBL]
@Test
public void testnegativeingabe(){
   DreiecksArt result = klassifiziereDreieck(3,-4,5);
   assertEquals("Pruefe ob es Schlechte negative Laengen erkennt", DriecksArt.Normal, result);
}
  \end{lstlisting} &  Need to check if it can recognize invalid lengths that are negative. \\ 
 \hline
\begin{lstlisting}[language=java,frame=trBL]
@Test
public void testNaneingabe(){
   DreiecksArt result = klassifiziereDreieck(3,4,NAN);
   assertEquals("Pruefe ob es Schlechte Laengen erkennt die keine Zahlen sind", DriecksArt.Normal, result);
}
  \end{lstlisting} &  Need to check if it can recognize lengths that are not Numbers. \\ 
 \hline
\begin{lstlisting}[language=java,frame=trBL]
@Test
public void testNaneingabe(){
   DreiecksArt result = klassifiziereDreieck(3,null,5);
   assertEquals("Pruefe ob es Schlechte Laengen erkennt die keine Werte haben", DriecksArt.Normal, result);
}
  \end{lstlisting} &  Need to check if it can recognize lengths that are null. \\ 
\hline
  \begin{lstlisting}[language=java,frame=trBL]

@Test
public void testgleichseitig(){
   DreiecksArt result = klassifiziereDreieck(null,null,null);
   assertEquals("Pruefe ob es Gleichseitigkeit bei Falscheingabe erkennt", DriecksArt.Normal, result);
}
  \end{lstlisting} text& Need to check if it will recognize equallateral triangles even if it has bad input. \\
\end{tabular}
\end{center}

\item There are 4 branches in the code:
\begin{list}
\item if all if-statements are false it executes int quad1 = seite1 * seite1; , int quad2 = seite2 * seite2; , int quad3 = seite3 * seite3; and return DreieckArt.Normal; 
\item if all sides are the same length it executes return DreieckArt.Gleichseitig;
\item if two sides are the same length it executes return DreieckArt.Gleichschenklig;
\item if the triangle is a right triangle it executes return DreieckArt.Rechtwinklig; after executing int quad1 = seite1 * seite1; , int quad2 = seite2 * seite2; and int quad3 = seite3 * seite3;
\end{list}

\begin{center}
\begin{tabular}{ | c | c | } 
  \hline
  \begin{lstlisting}[language=java,frame=trBL]

@Test
public void testgleichseitig(){
   DreiecksArt result = klassifiziereDreieck(4,4,4);
   assertEquals("Pruefe ob es Gleichseitigkeit erkennt", DriecksArt.Gliechseitig, result);
}
  \end{lstlisting} text& testing branch 2 \\ 
\hline
  \begin{lstlisting}[language=java,frame=trBL]
@Test
public void testrechtwinklig(){
   DreiecksArt result = klassifiziereDreieck(3,4,5);
   assertEquals("Pruefe ob es Rechtwinkligkeit erkennt", DriecksArt.Rechtwinklig, result);
}
  \end{lstlisting} & Testing Branch 4 \\ 
  \hline
  \begin{lstlisting}[language=java,frame=trBL]
@Test
public void testgleichschenklig(){
   DreiecksArt result = klassifiziereDreieck(4,4,5);
   assertEquals("Pruefe ob es Gleichschenkligkeit erkennt", DriecksArt.Gleichschenklig, result);
}
  \end{lstlisting} &  Testing branch 3 \\ 
  \hline
  \begin{lstlisting}[language=java,frame=trBL]
@Test
public void testgleichschenklig(){
   DreiecksArt result = klassifiziereDreieck(2,4,5);
   assertEquals("Pruefe ob es Gleichschenkligkeit erkennt", DriecksArt.Gleichschenklig, result);
}
  \end{lstlisting} &  Testing branch 1 \\
\end{tabular}
\end{center}

\item The Programm all of the tests successfully, execpt the one that checks what it returns if all lenghts are bad values but the same type of bad value and the one that checks if it recognize negative lengths correctly. The Program doesn't handel bad inputs very well, it returns that a triangle is normal even when one of the inputs is null, Nan or 0 for a side of the triangel, or if all the inputs are the same even if they are bad values it will return that the triangle is equallateral or if two sides have the same bad value it will return that the triangle is an isosceles triangle which could be misleading and even if the sides of the triangel are negative it will still recognize a right triangle.

\item Yes, The Coverage criteria C0 and C1 were achieved by the tests because all Code was executed once and all conditions were true and false at least once. The Coverage criteria C1 and C0 don't seem very suitable for this type of problem because the code doesn't handel all possible conditions with conditional statements.
\item The structure test also don't seem very suitable because the code didn't anticipate all types inputs it could get. The Functionstest on the otherhand was suitable because it did cover all possible inputs that the code needs to be able to handel.
\item If null and Nan are the inputs for 2 sides e.g klassifiziereDreieck(Nan,null,5) it is likely that the program will crash when it tries to do arithmetic on these values.

\end{parlist}

\chapter{Task 3}
\begin{parlist}
	\item When working with multiple people on a bigger Software project, role assignment is crucial. Often the developers team gets to be split into to part, the part that does the actual developing and the testing team. The testing team develops automated test cases to adapt to the code keeping up with the other team every step of the developing process. Actuating a sort of "divide and conquer" process. Testing your own code would mean losing this "divide and conquer" principle so that the developing process accompanied by the testing process would rather take double the time. Moreover developers often carry on the "how to make this" mindset, when instead a tester often uses the mindset: "how to break this". Developers are not that much closer to the client so when it comes to properly and finally test a software someone else much more closer to the client, like the QA team, should instead test the Software. This rules makes even more sense if we consider that often times developers cannot detach themselves from the idea of how they coded the software, and therefore would be bias on how they would test the code (testing the code exactly how it was developed). \cite{stackexchangeDevelopersTest}\cite{Sommerville:2004aa}
	\item Test Driven Development (TDD) is the process of interleaving development with test. You develop the code incrementally, along with a set of tests for that increment. You don’t start working on the next increment until the code that you have developed passes all of its tests. Test-driven development was introduced as part of the XP agile development method. However, it has now gained mainstream acceptance and may be used in both agile and plan-based processes. There are four fundamentals steps of TDD:
		\begin{enumerate}
			\item Identification of the increment of functionality required
			\item Writing the test for said incremental functionality and implement it into an automated test 
			\item Running the test to fail to verify the branch reachability.
			\item Implementing the function for which the test was developed and run the test.
			\item Repeat for the next functionality. 
		\end{enumerate}
		TDD helps the developers to change hats on every step of the process, since the testing gets implemented before the actual coding is done, the bias of the developer is almost null and therefore force the teams to think both ways. Also TDD help decrease the phenomenon of Regression Testing, in which the functionalities before implemented have to be tested again when the new functionalities get implemented into the project, thus reducing the amount of time for testing in general. It is also claimed that use of TDD encourages better structuring of a program and improved code quality. \cite{Sommerville:2004aa}
\end{parlist}
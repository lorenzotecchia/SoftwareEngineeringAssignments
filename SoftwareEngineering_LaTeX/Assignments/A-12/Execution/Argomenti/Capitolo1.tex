\chapter{Task 1}
\begin{parlist}
	\item In the waterfall process model the phases of building the software system are often separated and the cascading use for the next phase to rely on the previous one give the name to the process model. In contrast the process model "incremental" gives an interleaved approach between phases to the software building; the system is developed as a series of version and each version release goes through normal phases of software building and each version implements new functionalities to the previous one. Finally the evolutionary model, fuses the model of "iterative" and "incremental" to first construct what could be described as the skeleton of the software system; in this model the processes of creating the software and maintain the software "collide" and the software evolves in response to new requirements with the passage of time, is especially used in large projects. \cite{Sommerville2015}\cite{unstopEvolutionaryModel}
	\item These are the choices:
		\begin{enumerate}
			\item The evolutionary model, since a "skeleton" of the concept of the system already exists, the paper one, the latter has to be used as a stepping stone to evolve into the digital and interactive version of it.
			\item Waterfall process model, since in this case the requirements don't really much evolve during the building of the system, the ABS like its control unit should serve just one and only one purpose, and also the whole system should be built in its entire functionality before starting any other process forward (e.g. V\&V). Falls into the more traditional mechanical process of building something.
			\item Incremental model, since the version of for example the KVV can be released and students that use it can give back valuable feedback about it and help the system to adapt new requirements and possible defects present into the system.\cite{Sommerville2015}
		\end{enumerate}
\end{parlist}
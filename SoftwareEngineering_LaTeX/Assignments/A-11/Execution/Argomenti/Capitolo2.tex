\chapter{Task 2}
Having read the document, it's clear that the argument that Dijkstra puts forward goes in favour of abolishing the GOTO statement, portraying of it as an "errore prone structure" kind of picture. Even though has drown some inspiration for his statement, Dijkstra wants to underline his remark explaining throughly the dangers of the GOTO statement. 
From personal experience with the C programming language that in fact still has in use the GOTO statement I was told very early on my programming experience that it should be avoided at any cost, and the fact that the structured programming is what is being taught in Universities all around the world makes Dijkstra's claim and advice very convincing. Personally agreeing with it is an understatement. I firmly believe in it, and just trying to tinker with such statement has taught me first hand that should be avoid at any cost except maybe for the occasion of the ASSEMBLY language.
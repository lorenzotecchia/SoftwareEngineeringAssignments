\chapter{Task 2}
\begin{parlist}
	\item The architectural styles are the following:
	\begin{itemize}
		\item Client/Server architecture: Spotify(Any streaming software), it can be seen in the way that there is one copy of a song/album on a server that can be streamed to different clients via the app.
		\item Multi-layer architecture: ChatGPT, we can observe the layer architecture by the fact that a GUI appears to the user and the text inputed into the GUI prompt a calculation done and stored by a server.
		\item Event-based systems: Strava (Any sport tracking system), the system has to output different statistics to the user in correspondence of the event that happen (user does a km of running).
		\item Data flow networks: The Tor browser, uses the network routing to rout through an onion like layer of connections, the flowing of this connection is solely based on the input of the previous one thus ignoring the other connection traversed. 
		\item Web architecture: Microsoft Azure(any web based cloud computing server) is clear how the cloud and therefore the web provides services for for computing pourposes.
	\end{itemize}
	\item 
	\begin{enumerate}
		\item Real-time behaviour: Event-driven system style, or event-driven software architectures patterns, because they handle multiple events at ones, so users don't have to wait for one process to finish before starting the next. Thus the response time doesn't depend on whether a process has already been completed.\cite{infoworldBenefitsChallenges}
		\item High portability: Micro kernel styles, because in the style there is a main system that can run on the target operating system that contains everything the system needs to function and then it can run smaller plug in systems that are independent of each other. So once you have created one of these plugins you can use it directly on any system where the main system is already implemented.\cite{spiritofsoftSoftwareArchitecture}
	\end{enumerate}
\end{parlist}
\chapter{Task 2}
\begin{parlist}
\item A usecase diagram is an Summary/ overview of general Information regarding the Users and their interactions with a System and how the different use cases affect each other in contrast to a usecase which provides more specific and complete information about a particular interaction of a specific user with the system.
source: https://www.lucidchart.com/pages/de/uml-anwendungsfalldiagramm
Advantages of usecase diagrams in uml are:
- they provide a useful overview of the context and requirements of a system
- they are easy to understand
Disadvantages of use case diagrams are:
- They are not very detailed / don't contain allot of information
- They are Generalized so they might exclude important information like Failure cases
\item The actor model element in a use case diagram represents a user which interacts with the system, it can be a person organization, or external system.

\item A use case diagram contains usecases ovals which represent the different usecases for a user, Associations lines between usecases and actors with denote which actors are associated with a use case and System boundaries which define the scope of the system e.g where the system starts and ends.

\item Use cases can have following types of relations to each other in a usecase-diagram:<<includes>>
and <<extends>>. includes indicates that the behavior of a usecase is present in another usecase where as extends indicates that additional behavior of another use case should be added to a usecase possibly conditionally.

\end{parlist}
